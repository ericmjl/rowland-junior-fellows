\textbf{Introduction:} One goal of viral genomic surveillance is to use
a virus' sequence to predict how virulent and pathogenic it may be.
Unfortunately, \textbf{the biggest gap hampering this vision is a lack
of viral protein sequence variants paired with systematically measured
phenotype data}, on which computational models can be trained to learn
the sequence-function mapping. As a Rowland Junior Fellow, I aim to (1)
develop safe, scalable and standardized assays to systematically measure
viral protein variants, and (2) integrate the data generation pipeline
with new advances in machine learning to accurately and rapidly predict
of viral risk profiles from sequence. My long-term goal is to build a
dashboard that predicts, \emph{in silico}, viral risk from its component
proteins' sequence, without requiring any experimentation on whole live
viruses.

\textbf{(1) Viral phenotyping:} Health risk due to viral infection is
partially determined by host response, and partially determined by a
virus' component biochemical properties, which are in turn determined by
its component proteins. In order to address the missing link of virus
phenotype data, \textbf{we will systematically characterize virus
protein variants for their \emph{in vitro} biochemical properties.}

We will start by characterizing the influenza polymerase complex
variants' replication rate, and neuraminidase variants' drug resistance.
We will use DNA synthesis and assembly methods to synthesize protein
variants, and use robotics to scale our variant library generation and
measurement steps. We will also build new assays for systematically
measuring other influenza viral proteins' biochemistry, for example, by
leveraging our genetic systems to rapidly test other emerging pathogens.
Our long-term goal here is to develop a plug-and-play experimental
system for biochemically testing any new virus' proteins and its
variants within days of sequencing, for which the generated data could
help guide clinical treatment and epidemiological interventions. A
``stretch goal'' would be determining the minimum number of assays
required to reliably predict a virus' risk profile.

\textbf{(2) Computational prediction:} The viral phenotyping data
generated constitute a gold standard, densely measured dataset with
paired sequence and measurements. This data can be leveraged for
predicting viral protein activity from sequence. This problem can
essentially be cast as learning a non-linear mapping from genotype space
to phenotype space. \textbf{We will use supervised learning algorithms
to learn the mapping from genotype space to phenotype space, while also
partnering with current collaborators to develop new learning algorithms
for this task.}

With our experimental pipeline, data, and models, I aim to build an
integrated ``viral forecasting'' system capable of being updated with
new experimental data. Such a system may enable epidemiologists to
monitor and forecast which aspects of a virus' risk are increasing
(e.g.~vaccine suscepitibility vs.~drug susceptibility), and hence tailor
outbreak responses most effectively. It may also help guide clinical
decision making, such as recommending drug combinations that a patient's
virus population would be most sensitive to. With a fast experimental
pipeline infrastructure, our machine learning models could be refined in
real-time as new viral protein variants are continually tested.

\textbf{Funding Avenues:} In pursuit of these goals, I have written two
Broad\emph{Next10} grant applications, one of which was co-written with
colleagues at the Broad Institute, to develop such standardized, safe
and scalable assays for the influenza polymerase and neuraminidase.
\textbf{Both our grants were awarded, totalling \$80,000 in funds.}
Additionally, my current advisor and I are collaborating with the
Harvard Intelligent and Probabilistic Systems group on an NIH R21 grant
to fund these efforts further. Other planned funding sources include
DARPA's Prophecy program, philanthropic groups (Gates and Simons
foundations), and the NIH/NIAID. We will also explore data
access/licensing models with interested industry partners to enable our
research and engineering efforts to be self-sustaining.
